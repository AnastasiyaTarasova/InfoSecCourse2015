\documentclass[12pt,a4paper]{article}
\usepackage[utf8]{inputenc}
\usepackage[russian]{babel}
\usepackage{amsmath}
\usepackage{amsfonts}
\usepackage{amssymb}
\usepackage{graphics}
\usepackage[pdftex]{graphicx}
\usepackage{lscape}
\usepackage{listings}
\author{Анастасия Тарасова}
\title{Отчет по лабораторной работе №2 :\\ Утилита для исследования сети и сканер
портов Nmap}
\begin{document}
\maketitle
\section{Цель работы}

\section{Ход работы}
Определить набор и версии сервисов запущенных на компьютере в диапазоне адресов.
\newpage
\subsection{Поиск активных хостов}



\verb+db_nmap -sN 100.0.0/24+ - поиск активных хостов\\

\textbf{Вывод:}

\begin{lstlisting}
Starting Nmap 6.47 ( http://nmap.org ) at 2015-03-29 18:01 EDT
Nmap scan report for 100.0.0.24
Host is up (0.00020s latency).
Not shown: 977 closed ports
PORT     STATE         SERVICE
21/tcp   open|filtered ftp
22/tcp   open|filtered ssh
23/tcp   open|filtered telnet
25/tcp   open|filtered smtp
53/tcp   open|filtered domain
80/tcp   open|filtered http
111/tcp  open|filtered rpcbind
139/tcp  open|filtered netbios-ssn
445/tcp  open|filtered microsoft-ds
512/tcp  open|filtered exec
513/tcp  open|filtered login
514/tcp  open|filtered shell
1099/tcp open|filtered rmiregistry
1524/tcp open|filtered ingreslock
2049/tcp open|filtered nfs
2121/tcp open|filtered ccproxy-ftp
3306/tcp open|filtered mysql
5432/tcp open|filtered postgresql
5900/tcp open|filtered vnc
6000/tcp open|filtered X11
6667/tcp open|filtered irc
8009/tcp open|filtered ajp13
8180/tcp open|filtered unknown
MAC Address: 08:00:27:D4:D7:99 (Cadmus Computer Systems)

Nmap scan report for 100.0.0.23
Host is up (0.0000050s latency).
All 1000 scanned ports on 100.0.0.23 are closed

Nmap done: 256 IP addresses (2 hosts up) scanned in 31.31 seconds

\end{lstlisting}
\newpage
\subsection{Определить открытые порты}

\verb+db_nmap -sS 100.0.0/24+ - просмотр активных портов\\

\textbf{Вывод:}

\begin{lstlisting}
Starting Nmap 6.47 ( http://nmap.org ) at 2015-03-29 18:33 EDT
Nmap scan report for 100.0.0.24
Host is up (0.00011s latency).
Not shown: 977 closed ports
PORT     STATE SERVICE
21/tcp   open  ftp
22/tcp   open  ssh
23/tcp   open  telnet
25/tcp   open  smtp
53/tcp   open  domain
80/tcp   open  http
111/tcp  open  rpcbind
139/tcp  open  netbios-ssn
445/tcp  open  microsoft-ds
512/tcp  open  exec
513/tcp  open  login
514/tcp  open  shell
1099/tcp open  rmiregistry
1524/tcp open  ingreslock
2049/tcp open  nfs
2121/tcp open  ccproxy-ftp
3306/tcp open  mysql
5432/tcp open  postgresql
5900/tcp open  vnc
6000/tcp open  X11
6667/tcp open  irc
8009/tcp open  ajp13
8180/tcp open  unknown
MAC Address: 08:00:27:D4:D7:99 (Cadmus Computer Systems)

Nmap scan report for 100.0.0.23
Host is up (0.0000050s latency).
All 1000 scanned ports on 100.0.0.23 are closed

Nmap done: 256 IP addresses (2 hosts up) scanned in 29.82 seconds
\end{lstlisting}
\newpage
\subsection{Определить версии сервисов}
\verb+db_nmap -sV 100.0.0/24+ - показать версии сервисов\\

\textbf{Вывод:}

\begin{lstlisting}
Starting Nmap 6.47 ( http://nmap.org ) at 2015-03-29 18:51 EDT
Nmap scan report for 100.0.0.24
Host is up (0.00015s latency).
Not shown: 977 closed ports
PORT     STATE SERVICE     VERSION
21/tcp   open  ftp         vsftpd 2.3.4
22/tcp   open  ssh         OpenSSH 4.7p1 Debian 8ubuntu1 
(protocol 2.0)
23/tcp   open  telnet      Linux telnetd
25/tcp   open  smtp        Postfix smtpd
53/tcp   open  domain      ISC BIND 9.4.2
80/tcp   open  http        Apache httpd 2.2.8 ((Ubuntu) DAV/2)
111/tcp  open  rpcbind     2 (RPC #100000)
139/tcp  open  netbios-ssn Samba smbd 3.X (workgroup: WORKGROUP)
445/tcp  open  netbios-ssn Samba smbd 3.X (workgroup: WORKGROUP)
512/tcp  open  exec        netkit-rsh rexecd
513/tcp  open  login?
514/tcp  open  shell?
1099/tcp open  rmiregistry GNU Classpath grmiregistry
1524/tcp open  shell       Metasploitable root shell
2049/tcp open  nfs         2-4 (RPC #100003)
2121/tcp open  ftp         ProFTPD 1.3.1
3306/tcp open  mysql       MySQL 5.0.51a-3ubuntu5
5432/tcp open  postgresql  PostgreSQL DB 8.3.0 - 8.3.7
5900/tcp open  vnc         VNC (protocol 3.3)
6000/tcp open  X11         (access denied)
6667/tcp open  irc         Unreal ircd
8009/tcp open  ajp13       Apache Jserv (Protocol v1.3)
8180/tcp open  http        Apache Tomcat/Coyote JSP engine 1.1
1 service unrecognized despite returning data. If you know the 
service/version, please submit the following fingerprint at 
http://www.insecure.org/cgi-bin/servicefp-submit.cgi :
SF-Port514-TCP:V=6.47%I=7%D=3/29%Time=551881FD%P=i686-pc-linux-
gnu%r(NULL,
SF:33,"\x01getnameinfo:\x20Temporary\x20failure\x20in\x20name
\x20resolutio
SF:n\n");
MAC Address: 08:00:27:D4:D7:99 (Cadmus Computer Systems)
Service Info: Hosts:  metasploitable.localdomain, localhost, 
irc.Metasploitable.LAN; OSs: Unix, Linux; CPE: cpe:/o:linux:
linux_kernel

Nmap scan report for 100.0.0.23
Host is up (0.0000050s latency).
All 1000 scanned ports on 100.0.0.23 are closed

Service detection performed. Please report any incorrect results
at http://nmap.org/submit/ .
Nmap done: 256 IP addresses (2 hosts up) scanned in 39.96 seconds
\end{lstlisting}
\newpage
\subsection{Изучить файлы nmap-services, nmap-os-db, nmap-service-probes}

\subsection{Добавить новую сигнатуру службы в файл nmap-service-probes}
(для этого создать минимальный tcp server, добиться, чтобы присканировании nmap указывал для него название и версию)
\subsection{Сохранить вывод утилиты в формате xml}
\subsection{Исследовать различные этапы и режимы работы nmap с использованием утилиты Wireshark}
\verb+Просканировать виртуальную машину Metasploitable2 используя nmap_db +

\verb+из состава metasploit-framework. Выбрать пять записей из файла+ 

\verb+nmap-service-probes и описать их работу Выбрать один скрипт из +

\verb+состава Nmap и описать его работу+
\section{Выводы}
Для поиска активных хостов испольховался ключ \textbf{-sN}, \textbf{--sS}
\end{document}