\documentclass[12pt,a4paper]{article}
\usepackage[utf8]{inputenc}
\usepackage[russian]{babel}
\usepackage{amsmath}
\usepackage{amsfonts}
\usepackage{amssymb}
\usepackage{graphics}
\usepackage[pdftex]{graphicx}
\usepackage{lscape}
\author{Анастасия Тарасова}
\title{Отчет по лабораторной работе №2 :\\ Утилита для исследования сети и сканер
портов Nmap}
\begin{document}
\maketitle
\section{Цель работы}

\section{Ход работы}
Определить набор и версии сервисов запущенных на компьютере в диапазоне адресов
\subsection{Поиск активных хостов}

\subsection{Определить открытые порты}

\subsection{Определить версии сервисов}

\subsection{Изучить файлы nmap-services, nmap-os-db, nmap-service-probes}

\subsection{Добавить новую сигнатуру службы в файл nmap-service-probes}
(для этого создать минимальный tcp server, добиться, чтобы присканировании nmap указывал для него название и версию)
\subsection{Сохранить вывод утилиты в формате xml}
\subsection{Исследовать различные этапы и режимы работы nmap с использованием утилиты Wireshark}
\verb+Просканировать виртуальную машину Metasploitable2 используя nmap_db из состава metasploit-framework. Выбрать пять записей из файла nmap-service-probes и описать их работу Выбрать один скрипт из состава Nmap и описать его работу+
\section{Выводы}

\end{document}