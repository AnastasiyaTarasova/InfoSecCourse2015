\documentclass[12pt,a4paper]{article}
\usepackage[utf8]{inputenc}
\usepackage[russian]{babel}
\usepackage{amsmath}
\usepackage{amsfonts}
\usepackage{amssymb}
\usepackage{graphics}
\usepackage[pdftex]{graphicx}
\usepackage{lscape}
\usepackage{listings}
\author{Анастасия Тарасова}
\title{Отчет по лабораторной работе №3 :\\ Инструмент тестов на проникновение Metasploit}
\begin{document}
\maketitle
\section{Цель работы}

\section{Ход работы}

\newpage
\subsection{Подключиться к VNC-серверу, получить доступ к консоли}

\verb| Nmap: 5900/tcp open vnc|\\

\verb|search vnc|

\begin{lstlisting}

\end{lstlisting}

\verb|use auxiliary/scanner/vnc/vnc_none_auth |

\verb|auxiliary(vnc_none_auth) > info|\\

установим хосты которые будем сканироваеть
\verb|msf auxiliary(vnc_none_auth) > set rhosts 10.0.0.24|

\verb|rhosts => 10.0.0.24|

\verb|msf auxiliary(vnc_none_auth) > exploit|\\


\verb|msf auxiliary(vnc_none_auth) > set rhosts 10.0.0.22|

rhosts => 10.0.0.24

\verb|msf auxiliary(vnc_none_auth) > exploit|\\


\verb|msf auxiliary(vnc_none_auth) > set rhosts 10.0.0.22|

\verb|rhosts => 10.0.0.22|

\verb|msf auxiliary(vnc_none_auth) > exploit|

\verb|use auxiliary/scanner/vnc/vnc_login|

\verb|info|

\newpage
\subsection{Получить список директорий в общем доступе по протоколу SMB}

\subsection{Получить консоль используя уязвимость в vsftpd}

\subsection{Изучить файлы nmap-services, nmap-os-db, nmap-service-probes}

\subsection{Добавить новую сигнатуру службы в файл nmap-service-probes}

\subsection{Получить консоль используя уязвимость в irc}
\subsection{Armitage Hail Mary}

\section{Выводы}

\end{document}