\documentclass[12pt,]{article}
\usepackage[utf8]{inputenc}
\usepackage[russian]{babel}
\usepackage{amsmath}
\usepackage{amsfonts}
\usepackage{amssymb}
\usepackage{graphics}
\usepackage{lscape}
\author{Воробьев Олег и Тарасова Анасатсия}
\title{Аналитическое чтение тезисов с пленарных заседаний ACM CCS'13-14}
\begin{document}
\maketitle
\section{Cyber security}
Автор верит, что интернет рынок и интернет-культура воспитают в обществе основные нормы кибер-безопасности.
В наши дни все расширяющийся спектр интернет-ресурсов пораждает все большие и большие риски для правительств и бизнеса.
Все больше и больше средств и внимания уделяется противодействию кибер преступности и обеспечению сохранности своих данных.
\section{Next generation internet}
Интернет превосходит по своему развитию все ожидания. Его развитие обусловлено глубокой взаимосвязью между основными аспектами социума. Именно поэтому так важна проблема обеспечения безопасности.
Невозможно пресдказать как будет развиваться интернет даже в небольшие сроки, периодически выпускаемые обновления основаны только на текущем положении вещей. Сами основы интернета не позволяют качественно обеспесить безопасность. Другая важная проблема- аутентификация пользователей в мировом пространстве. Чтобы избежать таких проблем, автор участвует в разработке интернета нового поколения, а архитектуре которого будут заложены безопасность и приватность. Исследуя современный интернет, он находит способы превратить его в интернет следующего поколения.
\section{Cyber Arms Race}
Когда в 90 года интернет стал общедоступным и всем стал доступен огромный онлайн мир. Многие стали не только смотреть, но и производить что-то для других. Политики же решили что это отличная возможность наблюдать за людьми. Два ,возможно, главных открытия для поколения- интернет и мобильные телефоны являются превосходными средвтвами для слежки. Со времени выступления Эдварда Сноудена интернет слежка стала популярной темой. Но слежка так же важна, так как пресечение торговли наркотиками и террористических атак- это важные вещи, которые тоже перешли в интернет. В прессе пишут, что речь идет и мониторинге и составлении досье на людей, которые известно, что невиновны. Но неприятное можно найти на каждого. Спецслужбы США имеют полное законное право наблюдать за иностранцами, но мы все являемся иностранцами в отношении США. То есть при сипользовании интернет-сервисов, расположенных в США за нами ведется слежка. Американцы заявли, что шпионажем знимаются все, однако США имеет в этом приемущество так как большинство интернет-сервисов, таких как поисковые системы, облачные сервера, социальные сети находятся в США, так что наблюдение получается несимметричным. Почему бы не перестать пользоваться интернет-сервисами США? На практике сложно перестать пользоваться сервисами такими как Google, Facefook, Skype, Android, ICloud и т д. Однако, зачем беспокоиться, если вы не делаете ничего плохого? Автор считает, что вы должны быть возмущены вместо этого. Достижения в области науки и техники достигли того, что возможно наблюдение в больших количествах, но также возможна утечка.
\end{document}
