\documentclass[12pt,]{article}
\usepackage[utf8]{inputenc}
\usepackage[russian]{babel}
\usepackage{amsmath}
\usepackage{amsfonts}
\usepackage{amssymb}
\usepackage{graphics}
\usepackage{lscape}
\author{Тарасова Анасатсия}
\title{Аналитическое чтение тезисов с лекции 2 (от 16 фев 2015)}
\begin{document}
\maketitle
Исследовател безопасности пытались понять работу центра управления инцидентами (SOC) и как  аналитики в области безопасности выполняют свою работу. Эта работа обусловлена тем, что мониторинг и анализ безопасности - это не толькотехническая проблема. Исследователи должны принимать во внимание человеческий фактор. Большая работа в этом направлении благодаря интервью аналитиков безопасности в SOC. Но интервью не всегда возможно так как аналитики ограничены во времени и работают в стрессовой ситуации. Существует также вопрос доверия, который ограничевает количество информациии, получаемое при интетрвью. В своей работе Автор использует антропологческий подход к решению этой проблемы, проводя интервью со студентами, выполняющими работу аналитиков.
\section{Введение}
Целью исследования является получение целостного взгляда на работу центра упрравления инцидентами и  выделить следубщие аспекты:
\begin{itemize}
\item Структура команда общих и учебных центрах.
\item Обучение методологии для новых аналитиков
\item Оперативные рабочие потоки
\item Использование средств безопасности и программного обеспечения
\item График работы смен аналитиков
\item Метрики, используемые для оценки эффективности SOC
\end{itemize}
С указанными выше ценлями был принят антропологический подход к решению проблемы, в котором участвуют студенты компьютерных наук как аналитки в области безопасности академических и корпоративных SOC. Целью является понимание рботающего окружения с токи зрения аналитика.

Два центра, говорится далее Corp1-SOC и Corp2-SOC, относятся к двум корпорациям, которые предлагают информационо-технические услуги, со штаб-квартирок в США. Третий SOC,  далее U-SOC, - операционный центр при общественном университете в США. Один студент прикреплен в каждый SOC как аналитик безопасности в течение двух месяцев. Студенты обучаются выполнять действующие задачи аналитиков. Студенты документируют свои ежедневные наблюдения. Записи периодически анализируются антропологом, профессором Майклом Уэском (университет штата Канзас). Таким образом работники играют роль как аналитиков SOC, так и иссле исследователей, осуществляющих критику на своем опыте и наблюдении в SOC.

\section{Работа}
Суть работы в том, что работники на метсе пытаются получить перспективу SOC отличную от аналитика (родной) точки зрения. Родная точка зрения включает в себя неявное знание в наблюдаемом SOC которое может быть получено только через обучение, как мы делаем в нашей работе.

\section{Тренинг}

Каждый из работников, который нходится на месте под наблюдением профессора Майкла Уэска.
Двое из работников прошли курс антропологии профессора Уэска в Университете штата Канзас. Третий исследователь получил ускоренный курс антропологии профессора Уэска удаленно. Студенты должны делать замечания и записать только факты, а не мнения или предубеждения.
Каждый из студентов поддерживал цифровой журнал, где они документируют каждое событие и связь в SOC.

\section{CORPORATION-I (CORP-I) SOC}
Аспирант в области компьютерных наук работает аналитиком первого уровня в течение двух месяцев. Корпорация является многонациональной компанией, с филиалами во всем мире.
\subsection{Команды}
В SOC каждая команда работает во главе с менеджером, и все команды во главе с одним менеджером.
\subsubsection{Управление}
Оперативная группа состоит из  аналитиков двух уровней во главе с менеджером операций. Управление осуществляется 24 часа в сутки и 365 дней в году.Оперативная группа в настоящее время состоит из
20 аналитиков первого уровня и 2 L2 аналитиков второго уроовня.  Каждый аналитик работает 4 дня в неделю по 10 часа в сутки. Существуют три смены каждый день, утренние, дневные, и ночные смены. Сдвиги спланированы таким образом, что существует по крайней мере, 2 аналитика первого уровня в каждой смене.
\subsubsection{Engineering}

\end{document}