\documentclass[12pt,]{article}
\usepackage[utf8]{inputenc}
\usepackage[russian]{babel}
\usepackage{amsmath}
\usepackage{amsfonts}
\usepackage{amssymb}
\usepackage{graphics}
\usepackage{lscape}
\author{Тарасова Анасатсия}
\title{Эссе №2}
\begin{document}
\maketitle
\section{Введение}
Нв сегодняшний день растет потребность в  области информационной безопасности работников, особенно в компюерной сети. Однако работники не решаются делиться информацией для исследований с посторонними, ссылаясь на нехватку времени и загруженность, и это является наиболее огорчающим фактором.Цель исследования заключалась в разработке понимания правительства NOC в качестве основы для будущей сети исследований понимания ситуации. Но тем не менее, ориентированное на человека, исследование информационной безопасности работников имеет ряд проблем. В этой статье Автор описывает свои соображения, допущенные при разработки плана исследований, и уроки, извлеченные при проведении исследований в NOC.
\section{Практический пример}
Средой назначеня был операционный центр, отвечающий за безопасность и защиту крупной государственной сети. Доступ к информации был только у работников этажа. Принимая во внимание методику и способы исследования, Автор остановил вывод на этнографическом подходе. Из-за сложной исследовательской среды, мне нужно методов, которые были гибкими с минимальным воздействием на окружение. Встречи проводились 12 месяцев. 
\subsection{Встречи}
Первые встречи были направлены на сбор информации для минимизации исследований н окружающиюу среду, если можно так выразиться. Семь интервью были проведены с мужчинами, которые имели опыт работы с или в NOC. Длились интервью от 45 минут до 1,5 часов. Первые четые интервью были проведены с суррогатных пользователей.Автор использовал этоттермин для описания людей, которые похожи целевых пользователь общими знаниями или опытом. Их используют, когда целевые пользователи сильно загружены или недоступны. Это очень часто бывает с загруженной сетью оборонных аналитиков. Важно, что суррогатные пользователи не являются целевыми пользователями и принимают риск, используя свои собственные предубеждения на представление о том, что может сделать и подусвть целевой пользователь. Первые два интервью были с исследователями в научно-исследовательской организацией, связанной с родительской агентства NOC.Исследователи они были в состоянии предоставить общую информацию о NOC. Также были проведены интервью с аналитиками и менеджерами NOC, цель их организации стояла в обеспечении организации технологии аналитической поддержки. Последние три интервью были с людьми, которые на самом деле работали в  NOC. Последнее интервтю было с менеджером NOC, который смог предоставить Автору доступ к NOC по мере необходимости.

\subsection{Наблюдения}
Наблюдения происходили в основном во время ночной смены. Автор наблюдал за связью между менеджерами и аналитиками, за запланированными втречами, за моделированием кибер-событий. Автор присутствовал на учениях, предназначенных для тестирования новых страегий. Эти наблюдения дали мне ценную информацию для НОК операций, сильные и слабые стороны, и потенциальные области для будущего исследования.
\subsection{Card Sorting}
Card sorting провели с 12 аналитиками и менеджерами (все мужчины) с использованием 44 кибер ситуаций. Каждый вопрос был написан на карточке. Вопрос были получены из интервью и наблюдений.Результаты этого исследования дали результаты, подтвержающие, что связанные с кибер-аналитической работой аналитики и менеджеры чувствуют себя наиболее важными в вопросах информирования ситуации.   
\section{Discussion}
Проведение исследований, сосредоточенных на человеке, в сетевых операционных центрах является проблемой.Интенсивность работы аналитиков, чувствительность информационной среды, а также доступ к среде и люди создают барьеры для ведения исследований, сосредоточенных на человеке, в NOC.Ниже приводится краткое изложение основных выводов из этого исследования, как результаты этого исследования были использованы до настоящего времени, и извлеченные уроки, которые могли бы способствовать разработке будущих исследований в этой области.
\subsection{Главные выводы}
Наиболее сложный аспект работы NOC менеджера заключался в поддержании "миссии на уровне" осведомленности о состоянии сети. NOC является эффективным при своей миссии где менеджеры видят значительные улучшения, которые могут быть достигнуты благодаря использованию новых инструментов или артефактов.
\subsection{Применение результатов}
На сегодняшний день Автор использовал результаты этого исследования в нескольких направлениях. Ушлубленный анализ исследования card sorting привел к систематизированию в вопросах кибер обстановки, что  может быть использовано в разработке NOC, ориентированной на пользователя.
\subsection{Накопленный опыт}
Разрабатывается план исследований, который выполнится в течение длительного периода времени. Одного визита недостаточно, и через несколько часов недостаточно. Необходимо регулярно и часто посещать NOC. Эир минимизирует воздействие на деятельность и обеспечит долгосрочную перспективу. Посещать достаточно часто чтобы поддерживать отношение, но не так часто чтобы отвлекать. Можно выбрать методологию исследования с низким воздействием на окужающую средц. Несколько больших исследований могут быть сделаны в течение долгого времени и корректироваться по сере результатов развития научно-исследовательского проекта. Слишком дружественные отношения могут породить конфликт интересов и оказать сильное влияние на исследование.
\end{document}
