\documentclass[12pt,]{article}
\usepackage[utf8]{inputenc}
\usepackage[russian]{babel}
\usepackage{amsmath}
\usepackage{amsfonts}
\usepackage{amssymb}
\usepackage{graphics}
\usepackage{lscape}
\author{Тарасова Анасатсия}
\title{Эссе №2}
\begin{document}
\maketitle
\section{Введение}
Нв сегодняшний день растет потребность в  области информационной безопасности работников, особенно в компюерной сети. Однако работники не решаются делиться информацией для исследований с посторонними, ссылаясь на нехватку времени и загруженность, и это является наиболее огорчающим фактором.Цель исследования заключалась в разработке понимания правительства NOC в качестве основы для будущей сети исследований понимания ситуации. Но тем не менее, ориентированное на человека, исследование информационной безопасности работников имеет ряд проблем. В этой статье Автор описывает свои соображения, допущенные при разработки плана исследований, и уроки, извлеченные при проведении исследований в NOC.
\section{Практический пример}
Средой назначеня был операционный центр, отвечающий за безопасность и защиту крупной государственной сети. Доступ к информации был только у работников этажа. Принимая во внимание методику и способы исследования, Автор остановил вывод на этнографическом подходе. Из-за сложной исследовательской среды, мне нужно методов, которые были гибкими с минимальным воздействием на окружение. Встречи проводились 12 месяцев. 
\subsection{Встречи}
Первые встречи были направлены на сбор информации для минимизации исследований н окружающиюу среду, если можно так выразиться. Семь интервью были проведены с мужчинами, которые имели опыт работы с или в NOC. Длились интервью от 45 минут до 1,5 часов. Первые четые интервью были проведены с суррогатных пользователей.Автор использовал этоттермин для описания людей, которые похожи целевых пользователь общими знаниями или опытом. Их используют, когда целевые пользователи сильно загружены или недоступны. Это очень часто бывает с загруженной сетью оборонных аналитиков. Важно, что суррогатные пользователи не являются целевыми пользователями и принимают риск, используя свои собственные предубеждения на представление о том, что может сделать и подусвть целевой пользователь. Первые два интервью были с исследователями в научно-исследовательской организацией, связанной с родительской агентства NOC.Знания и опыт исследователей были ограничены проблемами они работали на, они были в состоянии предоставить общую информацию о NOC.

\section{Cyber Arms Race}

\end{document}